\documentclass[ngerman,a4paper]{scrartcl}
\usepackage[utf8]{inputenc}
\usepackage[T1]{fontenc}
\usepackage[left=1.50cm, right=1.50cm, top=1.50cm, bottom=2cm]{geometry}
\usepackage[ngerman]{babel}

\usepackage{graphicx}
\usepackage{tabularx}

\usepackage{newcode}
\usepackage{verbatim}


\author{Lucas Manuel Palomo Lauterbach}
\title {JSSTE Dokumentation \\ Version 1}
\begin{document}
	\maketitle
	\newpage
	\tableofcontents
	\newpage
	\section{Was ist JSSTE?}
JSSTE ist eine View-Library, die es ermöglicht Statisch aufgebaute HTML Seite jeweils mit anderem Inhalt zu füllen.



\section{Statische Variablen}

Statische Variablen sind zum festlegen der Page Eigenschaften da.\\
\newline
Liste aller Statischen Variablen.
\newline
\begin{tabularx}{\textwidth}{|c|X|X|}
	\hline
	\_TEMPLATE\_ & \inlinecode{JSSTE}{\"\_TEMPLATE\_\":\"startpage\",} & gibt an wie die Template-Datei heißt. wenn diese Variable nicht existiert oder keinen wert hat, wird diese nicht im Browser anzeigbar sein ( gut für Include-Dateien )  \\
	\hline
	\_STYLES\_ & \inlinecode{JSSTE}{\"\_STYLES\_\":[\"main.css\",\"index.css\"],} & ist eine Liste von CSS Dateien die man einbinden kann ( Optional ) \\
	\hline
	\_IMPORTS\_ & \inlinecode{JSSTE}{\"\_IMPORTS\_\":[\"./include1\",\"include2\"],} & ist eine Liste von jsste Dateien die man einbinden kann ( Optional ) \\
	\hline
\end{tabularx} 
\begin{comment}
\\
\begin{lstlisting}[style=JSSTE, caption={Beispiel einer Page}]
{
	"_TEMPLATE_":"startseite",
	"_STYLES_":["base.css"],
	"_INCLUDES_":["baseScript"]
}
\end{lstlisting}

\begin{lstlisting}[style=JSSTE, caption={Beispiel einer Inlcude-Datei}]
	{
		"_STYLES_":["base.css"],
		"_INCLUDES_":["baseScript"]
	}
\end{lstlisting}

\newline
\lstinputlisting[style=JSSTE]{./examples/pages/page2.jsste}
\end{comment}
\newpage
\section{Die Häufigsten Fehler}
\textit{SyntaxError: Unexpected string in JSON at position X}: 
\newline
Meist ist der Fehler darauf zurück zu führen, dass in der Page-Datei 
ein Komma vergessen wurde.
\begin{lstlisting}[style=JSSTE]
	{
	"_TEMPLATE_":"startseite",
	"_STYLES_":["base.css"]  <- Hier wäre der Fehler (Komma Fehlt)
	"meineVariable":"ich mag Kuchen"
	}
\end{lstlisting}

\textit{SyntaxError: Unexpected token \} in JSON at position X}: 
\newline
Dieser Fehler rührt meist daher, die letzte Variable in der Page-Datei nicht mit einem Komma abgeschlossen werden darf
\begin{lstlisting}[style=JSSTE]
{
"_TEMPLATE_":"startseite",
"_STYLES_":["base.css"],  
"meineVariable":"ich mag Kuchen", <- Hier wäre der Fehler (Komma zu viel)
}
\end{lstlisting}

\textit{SyntaxError: Unexpected token X in JSON at position Y}: 
\newline
Dieser Fehler rührt meist daher, das beim erstellen einer Variable die Anführungszeichen vergessen wurden.
\begin{lstlisting}[style=JSSTE]
{
"_TEMPLATE_":"startseite",
"_STYLES_":["base.css"],  
meineVariable:"ich mag Kuchen" <- Hier wäre der Fehler (Variable steht nicht in Anführungszeichen)
}
\end{lstlisting}







	

\end{document}